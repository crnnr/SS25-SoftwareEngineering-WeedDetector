\documentclass[12pt]{article}
\usepackage[margin=1in]{geometry}
\usepackage{booktabs}
\usepackage{amsmath}
\usepackage{array}
\usepackage[T1]{fontenc}
\usepackage{lmodern}

\begin{document}

\title{Tool Auswahl}
\date{15. Mai 2025}

\maketitle

\section{Einleitung}
In diese sektion wird eine ausführlichere übersicht zu verschiednen Tools und Technolgien gegeben. Wir beleuchten die Hauptmerkmale, Stärken und schäche von objekterkennungssystemen in unterschiedlichen Anwendungsszenarien. Die Reihenfolge der vorgestellten Methoden spiegelt weder eine Wertung noch eine empfehlung wieder.

\section{Methodik}
Die bewertung basiert auf Literaturrecherche und experimemtellen Benchmarks. Wir haben mehrere Datensätze genutzt, um Genauigkeit und Geschwindikeit zu testen. Die Komplexität wird anhand der erforderlichen Konfigurationsschritte und Abhängigkeiten beurteilt. Dabei floss sowohl Codeaufwand als auch Dokumentationsqualität mit ein. Jede Methode wurde auf einer identischen Hardware getestet, um die Ergebnisse besser vergleichbar zu machen.

\section{Vergleichmatrix}
\begin{center}
\begin{tabular}{lccc}
\toprule
Tool & Genauigkeit & Geschwindikeit & Komplexität \\
\midrule
Faster R-CNN & Hoch & Mittel & Hoch \\
SSD & Mittel & Hoch & Mittel \\
RetinaNet & Hoch & Mittel & Hoch \\
YOLO & Mittel-Hoch & Sehr Hoch & Niedrig \\
Detectron2 & Hoch & Niedrig & Hoch \\
\bottomrule
\end{tabular}
\end{center}
Diese matrix zeigt, dass SSD und YOLO in der Geschwindikeit punkten, während klassische region-proposal-methoden wie Faster R-CNN und RetinaNet bei Genauigkeit vorne liegen.

\newpage

\section{Ergebnisse und Diskussion}
Die experimente ergaben, dass YOLO in den meisten Fällen ein sehr gutes Verhältniss aus Geschwindigkeit und Genauigkeit bietet. SSD war ähnlich schnell, lieferte aber etwas schlechtere Detektionen bei kleinen Objekten. Faster R-CNN erreichte oft höhere präzision, benötigte aber deutlich mehr rechenzeit. Detectron2 war robust, aber durch viele Abhängigkeiten und Konfigurationen schwerer einzurichten. RetinaNet zeigte gute Balance, war aber in Training und Deployment komplexer. Die praxis tests bestätigten, dass für Anwendungen mit Echtzeit-Anforderungen YOLO bevorzugt wird.

\section{Fazit}
Aus den oberen matrix es is klar das YOLO in performance und einfacher implementazion punktet und damit die beste wahl ist. Für Projekte mit hohen Echtzeit-Anforderungen und begrenzten Ressourcen lffnen sich kaum Alternativen.  

\end{document}
