\documentclass[12pt]{scrreprt}
\usepackage[a4paper,margin=2.5cm]{geometry}
\usepackage[hidelinks]{hyperref}
\usepackage{enumitem}
\usepackage{graphicx}
\usepackage{listings}
\usepackage{xcolor}
\usepackage{tikz}
\usetikzlibrary{positioning, arrows.meta, shapes, fit}

% Define colors for code listings
\definecolor{codegreen}{rgb}{0,0.6,0}
\definecolor{codegray}{rgb}{0.5,0.5,0.5}
\definecolor{codepurple}{rgb}{0.58,0,0.82}
\definecolor{backcolour}{rgb}{0.95,0.95,0.92}

% Code listing style
\lstdefinestyle{mystyle}{
    backgroundcolor=\color{backcolour},   
    commentstyle=\color{codegreen},
    keywordstyle=\color{blue},
    stringstyle=\color{codepurple},
    basicstyle=\ttfamily\footnotesize,
    breaklines=true,                 
    keepspaces=true,                 
    numbers=left,                    
    numbersep=5pt,                  
    showspaces=false,                
    showstringspaces=false,
    showtabs=false,                  
    tabsize=2
}
\lstset{style=mystyle}

% Variables
\newcommand{\faculty}{Faculty Applied Information Technology}
\newcommand{\studies}{Bachelor of Cyber-Security}
\newcommand{\thesistitleDE}{Projektimplementierung \\ Reflexion zum Projekt "WeedDetector"}
\newcommand{\submissiondate}{11.\ Mai 2025}
\newcommand{\supervisor}{Prof.\ Dr.\ Holger Jehle}

\begin{document}
\begin{titlepage}
  \begin{center}
    {\Large Technische Hochschule Deggendorf\\
    \faculty\par}
    \vspace{0.2cm}
    {\Large Studiengang \studies\\}
    \vspace{2\baselineskip}
    {\Huge\bfseries \thesistitleDE\par}
    \vspace{3\baselineskip}
  \end{center}

  \vfill

  \parbox[t]{0.4\textwidth}{
    \textbf{Vorgelegt von:}\\[0.5em]
    Christof Renner\\
    Matrikelnummer: 22301943\\
    Manuel Friedl\\
    Matrikelnummer: 1236626\\[1em]
    Am: \submissiondate\par
  }
  \hfill
  \parbox[t]{0.4\textwidth}{
    \textbf{Prüfungsleitung:}\\[0.5em]
    \supervisor
  }
\end{titlepage}

\tableofcontents

\chapter{Controller-Implementierung}

Die WeedDetector-Anwendung folgt dem Model-View-Controller (MVC) Entwurfsmuster, wobei der \texttt{WeedDetectorController} als zentraler Koordinator zwischen GUI und Erkennungsmodell fungiert.

Die Hauptaufgaben des Controllers umfassen:
\begin{itemize}
    \item Ereignisbasierte Kommunikation zwischen GUI und Modell
    \item Verarbeitung von Benutzeraktionen (Bildauswahl, Erkennung, Robotersteuerung)
    \item Implementierung robuster Fehlerbehandlungsmechanismen
\end{itemize}

Die Kommunikation erfolgt über Callback-Funktionen, die eine lose Kopplung der Komponenten ermöglichen:

\begin{lstlisting}[language=Python, caption=Callback-Registrierung]
# GUI-Callbacks werden im Konstruktor registriert
self.gui.on_select_image = self.handle_select_image
self.gui.on_detect = self.handle_detect
self.gui.on_start_robot = self.handle_start_robot
self.gui.on_stop_robot = self.handle_stop_robot
\end{lstlisting}

\chapter{Model \& Modeltraining}

Der WeedDetector nutzt ein YOLO-Modell für die Unkrauterkennung aufgrund seiner exzellenten Balance aus Geschwindigkeit und Genauigkeit. Die Entscheidung basiert auf einem umfassenden Vergleich verschiedener Objekterkennungsalgorithmen.

Die \texttt{WeedDetectorModel}-Klasse implementiert:
\begin{itemize}
    \item Automatische Modellsuche mit Fallback-Mechanismus
    \item Konfigurierbare Erkennungsparameter (Konfidenz, IoU)
    \item Methoden für Training, Inferenz und Ergebnisvisualisierung
\end{itemize}

Der Trainingsprozess nutzt die Ultralytics-YOLO-API und einen umfangreichen Datensatz mit über 13.000 annotierten Bildern:

\begin{lstlisting}[language=Python, caption=Training]
def train(self, train_data_path, epochs=50):
    """Train the YOLO model with provided training data."""
    self.model.train(data=train_data_path, epochs=epochs)
\end{lstlisting}

Die Trainingsdaten wurden mit verschiedenen Augmentierungstechniken aufbereitet:
\begin{itemize}
    \item Horizontale Spiegelung (50\% Wahrscheinlichkeit)
    \item Zufällige Rotation und Bildausschnitte
    \item Belichtungsanpassungen
\end{itemize}

\chapter{Security Scanning \& Linting}

\section{Security Scanning mit Bandit}

Für kontinuierliche Sicherheitsüberprüfungen wird Bandit eingesetzt, ein auf Python spezialisiertes Security-Scanning-Tool. Die Integration erfolgt über GitHub Actions:

\begin{lstlisting}[language= Bash]
name: Security Scan
on:
  push:
    branches: [ main ]
  pull_request:
    branches: [ main ]
jobs:
  bandit:
    name: Bandit Scan
    runs-on: ubuntu-latest
    steps:
      - uses: actions/checkout@v3
      - uses: actions/setup-python@v4
        with: { python-version: '3.x' }
      - run: pip install bandit
      - run: bandit -r . --skip trojansource
\end{lstlisting}

Bandit analysiert den Code auf Sicherheitsprobleme wie Injektionsanfälligkeiten, Hartcodierte Credentials und Unsichere Funktionsaufrufe


\section{Codequalität mit Pylint}

Zur Sicherstellung konsistenter Codequalität wird Pylint als statisches Analysetool eingesetzt:

\begin{lstlisting}[language= Bash]
name: Pylint
on: [push]
jobs:
  build:
    runs-on: ubuntu-latest
    strategy:
      matrix: { python-version: ["3.13"] }
    steps:
    - uses: actions/checkout@v4
    - uses: actions/setup-python@v5
      with: { python-version: ${{ matrix.python-version }} }
    - run: pip install pylint
    - run: pylint $(git ls-files '*.py')
\end{lstlisting}

Pylint prüft PEP 8 Konformität, Dokumentationsqualität, Komplexitätsmetriken und potenzielle Bugs.

\chapter{Docker \& Deployment}

\section{Containerisierung}

Die Anwendung wird in Docker containerisiert, was eine konsistente Laufzeitumgebung sicherstellt. Die Dockerfile-Qualität wird mit Hadolint überprüft:

\begin{lstlisting}[language= Bash, caption=Docker Lint]
name: Docker Build and Test
on:
  push:
    paths: [ 'Dockerfile' ]
  pull_request:
    paths: [ 'Dockerfile' ]
jobs:
  lint-dockerfile:
    runs-on: ubuntu-latest
    steps:
      - uses: actions/checkout@v4
      - uses: hadolint/hadolint-action@v3.1.0
        with: { dockerfile: Dockerfile }
\end{lstlisting}

Der Container umfasst die Python-Laufzeitumgebung, YOLO- und OpenCV-Bibliotheken sowie die Anwendungskomponenten.

\section{Automatisiertes Deployment}

Das Deployment erfolgt automatisiert über GitHub Actions mit Integration in Docker Hub:

\begin{lstlisting}[language= Bash, caption=Build and Publish]
name: Build and Publish Docker image
on:
  push:
    branches: [ main ]
    paths: [ 'Dockerfile' ]
jobs:
  build-and-push:
    runs-on: ubuntu-latest
    steps:
      - uses: actions/checkout@v4
      - uses: docker/setup-buildx-action@v3
      - uses: docker/login-action@v3
        with:
          username: ${{ secrets.DOCKERHUB_USERNAME }}
          password: ${{ secrets.DOCKERHUB_TOKEN }}
      - uses: docker/build-push-action@v6
        with:
          context: .
          push: true
          tags: crnnr/weeddetector:latest
\end{lstlisting}

\section{Dependency-Management}

Dependabot übernimmt die automatische Aktualisierung von Abhängigkeiten für Python-Pakete, Docker-Images und GitHub Actions.

Das Projekt unterstützt verschiedene Deploymentstrategien:
\begin{itemize}
    \item Lokales Deployment für Entwicklung
    \item Container-basiertes Deployment für Produktion
    \item Versionierte Releases mit automatischen Updates
\end{itemize}

\end{document}